\chapter{návrh, výskum a postup}

\section{Ovládanie kurzora programom}
\tab[5 mm] Prvý cieľ, prenositeľnosť, sa podarilo u tohto prvého myľníka splniť výberom programovacích jazykov Python a Golang.
\subsection{Popis programu}
\tab[5 mm] Tento jednoduchý program má za úlohu \uv{iba} hýbať myšou pomocou klávesníc na klávesnici podľa preddefinovaného konfiguračného súbora. Tento jednodochý program je síce vytvorený v Golangu, ale dá sa pomerne ľahko integrovať do väčšieho projektu, ktorého programovacím jazykom je Python.
\subsection{Dôvod jeho vytvorenia}
\tab[5 mm] Súčasťou tejto diplomovej práce je aj hľadať alternatívne spôsoby riadenia počítačového zariadenia, preto bolo vhodné začať práve s týmto programom.
\section{Bitalino} 
\tab[5 mm] Do tohto zariadenia sme vkladali veľké očakávania, vzhľadom na jeho písané vlastnosti, no nepodarilo sa nám z neho získať zmysluplné dáte, čo súvisí aj s povahou DMO.
\subsection{Pokus}
\tab[5 mm] Bitalino bola naša prvá voľba, kvôli potenciálu, ktorý sľuboval. Pokúsili sme sa umiestniť elektródy, podľa návodov dokumentácie k tomuto zariadeniu.  a očakávali sme ...
\subsection{Problém}
\tab[5 mm] Kvôli zášklbom sa tažko pripevňovali elektródy na svoje miesta, navyše umiestnenie ako je na obr.\\
pridať obr. .......\\
sa len tažko dosahuje aj bez zášklbov, sme usúdili, že takéto zášklby by museli mať nejaké protiopatrenia (nehovoriac o tom, že aj v mozgu by ..., ale to by sa dalo odstrániť filtrami),%, rozpitvať to v diskusii?)
 preto sme sa rozhodli preskúmať iné technológie. 
\section{Vlastné riešenie(zariadenie)} % dať samostatnú kapitolu
\tab[5 mm] Pri preskúmaní technológii ako \uv{Tobii}, \uv{Eva faciál mouse} sa nám naskytla otázka, či tieto technológie by sa dali, zrýchliť a sprístupniť, pretože \uv{Tobii} je stredne nákladná technológia, ktorá nie vždy je voľná na predaj, navyše vyžaduje  veľkú pozornosť očí. % dopísať skúsenosti, pomalé, bolia oči, nemožnosť sa pohnúť ...
\\ 
\tab[5 mm] \uv{Eva faciál mouse} je síce voľne prístupná aplikácia pre všetky android zariadenia, ale je nepresná a pomalá.\\ 
\tab[5 mm] Preto sme si povedali, že urobíme program, ktorý skombinuje rozpoznávanie tváre podobné \uv{Eva faciál mouse} a joystickom, ktorý bude ovládaný jazykom a bude sa dávať do úst.\\ 
\tab[5 mm] preto sme zvolili využitie platformy Arduino, Bluetooth modulu a joysticku. Arduino, open-source mikrokontrolér, pktorý oskytuje flexibilitu a možnosť prispôsobenia pre rôzne projekty. Jeho schopnosť interakcie s rôznymi senzormi a aktuátormi umožňuje vytvárať robustné a prispôsobiteľné asistenčné riešenia.\\ 
\tab[5 mm] Bluetooth modul je kľúčovým komponentom nášho systému, umožňujúcim bezdrôtovú komunikáciu medzi Arduino a ďalšími zariadeniami. To nám poskytne väčšiu voľnosť pohybu a umožňuje vzdialené ovládanie asistenčných funkcií.\\ 
\tab[5 mm] Joystick, ako vstupný zariadenie, bude hrať kľúčovú úlohu pri poskytovaní presného ovládania, čo následne budeme kombinovať s algoritmami počítačového videnia. Joystick bude pripevnený na držadle, do kttorého sa zahryznú zuby. \\ 
\tab[5 mm] dopísať ......