\chapter*{Úvod} % chapter* je necislovana kapitola
 \addcontentsline{toc}{chapter}{Úvod} %rucne pridanie do obsahu
\markboth{Úvod}{Úvod} % vyriesenie hlaviciek

{ 
\tab[5mm] V dnešnej digitálnej dobe sú tradičné ovládacie metódy, ako sú klávesnica a myš, neodmysliteľnou súčasťou našej interakcie s počítačmi. Avšak, pre niektorých jednotlivcov postihnutých rôznymi fyzickými alebo neurologickými obmedzeniami môžu tieto konvenčné prostriedky predstavovať výzvu. Je preto nevyhnutné hľadať inovatívne a prispôsobené spôsoby ovládania, aby sme zabezpečili prístup k digitálnym technológiám pre všetkých.

\tab[5mm] Táto diplomová práca sa zameriava na štúdium využitia biosignálov pri vytváraní netradičných spôsobov ovládania elektrických zariadení a následne možné ovládanie virtuálneho vozíka. Biosignály predstavujú rôznorodú paletu fyzikálnych signálov generovaných živými organizmami, ktoré možno využiť na interakciu s počítačom. V rámci tohto štúdia bude špeciálny dôraz kladený na využitie Brain-Computer Interface (BCI) ako jedného z prístupov, ktorý môže poskytnúť vhodné biosignály pre netradičné ovládacie mechanizmy.

\tab[5mm] V nasledujúcich častiach práce sa však budeme zaoberať hlavným smerom nášho výskumu, a to využitím biosignálov na ovládanie pomocou jazyka. Analyzovať budeme nielen technické aspekty tejto problematiky, ale aj aplikácie a výhody, ktoré môže ponúknuť pre jednotlivcov s obmedzeniami. Cieľom bude preskúmať a vyvinúť inovatívne metódy, ktoré umožnia efektívne a presné ovládanie počítača prostredníctvom jazyka.
}

{ 
\tab[5mm] Začínajúci výskum \cite{s16111806} a \cite{app12178880} v oblasti biomedicíny ponúka zaujímavý pohľad na možnosť interakcie medzi osobou používajúcou invalidný vozík a počítačom s využitím umelej inteligencie (AI). Tieto štúdie naznačujú, že takáto interakcia by mohla byť realizovaná v reálnom čase, čím by sa otvárali nové perspektívy pre efektívne ovládanie vozíka.

}

{
\tab[5mm] Cieľom tejto diplomovej práce je identifikovať vhodné biosignály, prípadne ich kombinácie, ktoré by mohli slúžiť ako efektívne ovládacie mechanizmy pre invalidné vozíky. Kvôli technickým obmedzeniam práce s reálnymi vozíkmi a obmedzeniam v reálnom čase sa budeme zamerať na vývoj a testovanie týchto ovládacích mechanizmov na virtuálnych prostrediach s výstupom simulovaného ovládania myšou. Týmto spôsobom sa snažíme poskytnúť užitočné poznatky a návrhy, ktoré by mohli byť v budúcnosti implementované v reálnom prostredí.
}