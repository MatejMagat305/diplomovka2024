\chapter{Používateľské príručky}
\tab[5 mm] Tu sú popísané jednotlivé príručky k jednotlivým verziám.
\section{ Klávesnica - myš }
\tab[5 mm] 
Toto je softvér, ktorý má za cieľ umožniť ovládanie myši pomocou klávesnice. V priečinku \uv{bin} sa nachádza skompilovaný binárny súbor pre Linux a Windows (malo by byť možné spustiť aj na Macu, ale nemám Mac). V priečinku \uv{bin} je nastavenie pre Windows a v pobočke \uv{master} je nastavenie pre Linux. Ak program nenájde nastavenia, môžete ich sami prispôsobiť. V nastaveniach sú: 
\begin{itemize}
\item	Šípky na ovládanie pohybu myši
\item	 \uv{Enter},  pre ľavé kliknutie myšou 
\item	\uv{Delete}, pre pravé kliknutie myšou
\item	\uv{Q} pre ukončenie programu 
\item	\uv{D} pre deaktiváciu (nezabudnite, že to nebude pohybovať myšou)
\item	\uv{A} - pre opätovné aktivovanie
\end{itemize}

\section{GUI}
\tab[5 mm] toto je zatiaľ GUI pre vyzualizáciu interakciu ostatnách komponentov
\begin{itemize}
\item	\uv{w} pohyb hore
\item	\uv{s} pohyb dole
\item	\uv{a} pohyb doľava
\item	\uv{d} pohyb doprava 
\end{itemize}


\section{Face capture}
\tab[5 mm] Zapnúť aplikáciu a hýbať hlavou pred kamerou.


\section{hardware joystick}
\tab[5 mm] Zapnúť arduíno zasunúť prístroj do úst .

...