\chapter{Používateľské príručky}
\tab[5 mm] Tu sú popísané jednotlivé príručky k jednotlivým verziám.
\section{ Problém jedného telesa }
\tab[5 mm] Program má niekoľko atribútov, ktoré sú meniteľné počas chodu programu(behom módu init, ktorý je opísaný nižšie): 
\begin{itemize}
\item	\uv{AutoScale}, ktoré uvádza, či program má automaticky Škálovať.
\item	 \uv{FixedScale}, ktoré uvádza, že program nemá dovoliť Škálovať ani počítaču a ani pomocou klávesnice. 
\item	\uv{ForceCircularOrbit}, ktoré uvádza, či program má upraviť rýchlosť tak, aby dráha bola kruhová. 
\item	\uv{StepsForward} je hodnota krokov, ktoré má počítač vykonať bez animácie(na začiatku a aj na príkaz). 
\item	\uv{FixedCentre} je nastavenie na to aby stred “kamery” bol stále namierený do bodu 0,0.
\item	\uv{WriteExcelFiles} - nastavenie, či program má zapisovať údaje do excelovských súborov. 
\item	\uv{HowManyPositionsBack} - nastavenie, koľko pozícii do zadu si program má pamätať.
\item	 \uv{RememberWholeOrbit} - nastavenie, či si program má pamätať všetky pozície od začiatku.
\item	\uv{SpeedMovePixel} - slúži na určenie rýchlosti pohybu po obrazovke.
\end{itemize}

\tab[5 mm] Program má 4 módy(fázy):
\begin{itemize}
\item	\uv{init} - pomocou šípok  sa dá nastaviť krok a epsilon simulácie a iné premenné, užívateľ si tu môže vybrať atribút, ktorý chce zmeniť a následne ho môže zmeniť. Zobrazuje sa iba vybratá premenná, v tejto fáze sa objavujú rozličné varovania(warningy), ktoré upozorňujú na nenačítané dáta, alebo na únikovú rýchlosť   -  warning o rýchlosti sa objaví iba v prípade, že nie je nastavené  \uv{ForceCircularOrbit} na \uv{true}.
\item	\uv{animation–stop} -  epsilon sa meniť nedá, tak ako aj ďalšie atribúty. Animácia je stopnutá a ani výpočet sa nevykonáva automaticky, môže sa však nastaviť krok a krokovať .
\item	\uv{animation–play} - animácia beží sama.
\item	\uv{compute} - zobrazuje v polsekundových intervaloch koľko už počítač vypočítal, tento mód sa vykresľuje iba pre veľké \uv{StepsForward}, pre malé výpočet trvá tak krátko, že sa vykreslí na okamich čierna obrazovka, ale tá po polsekunde sa znova prekreslí novým stavom animácie.  
\end{itemize}
\tab[5 mm] Program sa štartuje dvojklikom - nie je nutné ho spúšťať cez príkazový riadok. Program automaticky načíta config.txt.\\
\tab[5 mm] Program z config.txt načíta okrem meniteľných atribútov aj také, ktoré sa v móde init nedajú meniť: \uv{PositionX}, \uv{PositionY}, \uv{VelocityX},\uv{VelocityY}, ktoré určujú začiatočnú polohu,  \uv{Scale}, ktorý označuje začiatočné škálovanie a \uv{SpeedScaledPercent}, ktoré určuje rýchlosť škáľovania v percentách, keď sa načíta príliš veľké percento (nad 100), tak program dosadí default(50) to isté pre nižšie ako 0.\\ 
\tab[5 mm] Počiatočný stav sa kontroluje v config.txt. program akceptuje riadky so správne zadaným menom, znamienkom rovnosti a hodnotami správneho typu.\\
\tab[5 mm] Ak program nerozpozná správne meno v riadku dosadí  oznámi to používateľovi po otvorení aplikácie hláskou \uv{warning}, ktorý riadok a čo bolo v tom riadku.  Program na začiatku vytvorí defaultné dáta, ktoré potom prepisuje načítaním z configu. \\
\tab[5 mm] Ak program načíta samé nuly dosadí default.\\
\tab[5 mm] POZOR: epsilon musí byť vo formáte .0xyz.... ak je epsilon väčšie ako 1.00(prípadne číslo menšie ako \(10^-13\)), tak sa epsilon nastaví na default.\\
\tab[5 mm] Hneď po spustení sa program spustí v \uv{init}-móde.\\
\tab[5 mm] Po potvrdení sa spustí mód \uv{compute}, ktorý vypočíta niekoľko krokov bez animácie, pričom počet krokov závisí na premennej \uv{StepsForward}. Po skončení \uv{compute} módu sa objaví animácia v móde \uv{animation–stop}.\\
\tab[5 mm] V \uv{animation–stop} a \uv{animation–play} sa pod animáciou zobrazuje pozícia x, y. nad animáciou sa zobrazujú informácie: ktorá metóda je to, zvolené epsilon, čas a krok.\\
\tab[5 mm] Animáciu(v  \uv{animation–stop} aj \uv{animation–play}) možno dočasne prerušiť úplne uvedením programu do výpočtového režimu(\uv{compute} módu), stlačením \uv{c} na klávesnici.
V   módoch  \uv{animation–stop} aj \uv{animation–play} sa dá meniť škálovanie, ale musia byť vypnuté dve funkcie(nastavené na \uv{false}):  \uv{FixedScale} a \uv{AutoScale}.\\
\tab[5 mm] V   módoch  \uv{animation–stop} aj \uv{animation–play} sa dá meniť stred \uv{kamery}, ale musí byť nastavený atribút \uv{FixedCentre} na \uv{false}.\\
\tab[5 mm] Ak je nastavená premenná \uv{WriteExcelFiles} na \uv{true} a zároveň \uv{RememberWholeOrbit}  je nastavená na \uv{true}, tak sa po skončení  ešte nejaký čas(závisí od veľkosti nazhromaždených údajov) bude zapisovať do xsxl súborov, ktorých názov pozostáva z názvu metódy a epsilonu, ktoré bolo zvolené v programe, súbory sa zapíšu do adresára files, keď program zistí, že taký adresár nie je, tak ho vytvorí.
Na ľavom boku animácie sa zobrazuje select voľba, ktorá slúži na to aby sa vedela vybrať plocha, ktorá sa \uv{pozametá} sprievodičom.
Na pravom boku je legenda \uv{pozametaných plôch} aj s hodnotami, ktorá sa objaví až po zmeraní prvej plochy.\\  
ovládanie:
\begin{itemize}

\item	Tlačidlo escape slúži na zatvorenie aplikácie.
\item	Tlačidlom enter na začiatku slúži na potvrdenie epsilónu a začatie simulácie. 
\item	Tlačidlom p sa spúšťajú/zastavujú všetky metódy do automatického módu.
\item	Medzerníkom sa dá nastaviť, či bývalé pozície telesa sa budú vykresľovať ako body alebo ako úsečky.
\item	Tlačidlami plus a mínus sa ovláda aký je veľký krok. 
\item	V móde \uv{init} sa tlačidlami doprava a doľava dá meniť parameter epsilón a iné atribúty, v móde  \uv{animation–stop} a \uv{animation–play} sa s týmito tlačidlami pohybuje "kamerou" (doprava a doľava). 
\item	V móde \uv{init} sa tlačidlami hore a dole vyberá, ktorý atribút užívateľ chce zmeniť, v móde  \uv{animation–stop} a “animation–play” sa s týmito tlačidlami pohybuje "kamerou"  (hore a dole).
\item	Tlačidlom e sa vymieňajú pohľady na metódy .
\item	Tlačidlom k sa krokuje animácia v zastavenom móde .
\item	Tlačidlom s sa vyberá pred stlačením k, ktoré plochy má rátať a ktoré nie.
\item	Tlačidlom c sa zastaví animácia, program sa prepne do režimu \uv{compute} a vykoná  \uv{StepsForward} krokov, potom sa tento režim vráti do režimu krokovania.
\item	Tlačidlami n a m sa ovláda škáľovanie. n vzďaľuje a m približuje.
\end{itemize}

\section{Problém dvoch telies bez retardácie}
